\pagestyle{myFancy}
\chapter{Yang-Mills Theories on the Lattice}
\section{The Yang-Mills Continuum Action}
The aim of this chapter is to discretize the Yang-Mills action on a hypercubic lattice in $4$ dimensions. In order to do so, the action is obtained firstly in the continuum, beginning from the simplest case, Quantum Electrodynamics.\\
This first section is based on material that can be found in standard textbooks on quantum field theory \cite{srednicki2007quantum, peskin1995introduction, weinberg1995quantum, kaku1993quantum, ramond1997field}, personal notes and computations.

\subsection{Scalar Fields}
In quantum field theory, a real scalar massive field is described, in a $4$-dimensional spacetime with metric $\eta_{\mu\nu} = \diag(-1,1,1,1)$, by the following Lorentz-covariant action (in natural units, where $c = \hslash = 1$):
\begin{equation}
    S[\phi] = \dV \pr{-\frac12 \partial^\mu\phi\partial_\mu\phi +\frac12 m^2\phi^2 +V(\phi)} \label{1:ActionScalar}
\end{equation}
where $V(\phi)$ is any potential, such as $\frac{g}{6}\phi^3$ or $\frac{g}{4!}\phi^4$.
Real scalar fields do not describe any real-world elementary particle, though they are useful to learn basic principles of quantum field theory, as they are the simplest fields that can be written.
%TODO: check notation consistency with lattice action after Wick rotation

\subsection{Dirac Spinor Fields}
Let us now take into consideration a (free) quantum field theory describing a fermion, such as a quark or a lepton. Its action can be written as:
\begin{equation}
      S_\psi[\psi(x),\psibar(x)] = \dV \left( \i\psibar\slashed\partial\psi - m\psibar\psi \right) \label{1:FreeQEDFermionAction}
\end{equation}
from which, upon the application of the variational principle, the Dirac equation follows:
\begin{equation}
    \left( \i\slashed\partial-m \right) \psi(x) = 0 \label{1:DiracEq}
\end{equation}
It can now be easily checked by direct computation that this action is invariant under a rigid (global) phase transformation, also called a global $U(1)$ transformation:
\begin{align*}
    \psi(x) &\rightarrow \psi'(x) = e^{-\i\alpha}\psi(x) \numthis\label{1:PhaseTransform} \\
    \psibar(x) &\rightarrow \psibar'(x) = \psibar(x) e^{\i\alpha}
\end{align*}
where $\alpha$ is a constant that does not depend on the spacetime coordinate $x$, because if $\alpha$ was a function of $x$, the kinetic term of the action \eqref{1:FreeQEDFermionAction} would not be invariant under such transformation.

\subsection{Quantum Electrodynamics\label{Sec:QED}}
As the free field theory itself is non interacting, it does not provide any real-world prediction, so it is useful to write an interacting action where the spinor field is coupled, for instance, to a vector field $A_\mu$, \ie the photon.
One way to implement this interaction is to ask for local, instead of global, invariance of the action \eqref{1:FreeQEDFermionAction} under the phase transformation \eqref{1:PhaseTransform}, where now $\alpha=\alpha(x)$.
In order to do so, the covariant derivative has to be defined as follows:
\begin{equation}
    D_\mu \equiv \partial_\mu+\i g A_\mu \label{1:QEDCovDeriv}
\end{equation}
where $g$ is the couplig constant.\footnote{Usually, in QED, $g$ is called $e$, the electron charge, though $g$ will be used in analogy to nonabelian gauge theories.}\\
The vector field's kinetic term is written in terms of its field-strength, namely:
\begin{align*}
    \Fmunu\equiv& -\frac{\i}{g}\left[D_\mu,D_\nu\right] = \\
    =& -\frac{\i}{g}\pr{D_\mu\pr{\partial_\nu+\i g A_\nu} - D_\nu\pr{\partial_\mu+\i g A_\mu}}=\\
    =& -\frac{\i}{g}\pr{\cancel{\partial_\mu\partial_\nu} + \i g\partial_\mu A_\nu -g^2A_\mu A_\nu - \cancel{\partial_\nu\partial_\mu} - \i g\partial_\nu A_\mu + g^2A_\nu A_\mu}=\\
    =& \partial_\mu A_\nu - \partial_\nu A_\mu + \i g\comm{A_\mu}{A_\nu}=\\
    =& \partial_\mu A_\nu - \partial_\nu A_\mu \numthis\label{1:FieldStrU1}
\end{align*}
where $\comm{A_\mu}{A_\nu}=A_\mu A_\nu-A_\nu A_\mu=0$ in the abelian theory.\\
Two different fields $A_\mu$ and $A_\mu'$ describe the same physics if one can be obtained from another through a gauge transformation:
\begin{align*}
    A_\mu'(x) =& A_\mu(x) + \frac1g \partial_\mu\alpha(x) \numthis\label{1:QEDGaugeTransf} \\
    \Fmunu' = \Fmunu + \frac1g&\left( \partial_\mu\partial_\nu-\partial_\nu\partial_\mu \right)\alpha(x) = \Fmunu
\end{align*}
Thus, the free action for the vector field is:
\begin{equation}
    S_{EM} = -\frac14\dV F_{\mu\nu}F^{\mu\nu} \label{1:FreeMaxwellAction}
\end{equation}
That is also gauge invariant, i.e. invariant under \eqref{1:QEDGaugeTransf}, as $\Fmunu$ is gauge invariant.\\
The term that broke the local phase invariance of the action \eqref{1:FreeQEDFermionAction} can now be ``absorbed'' by $A_\mu$ through a gauge transformation \eqref{1:QEDGaugeTransf}, thus making the full action gauge invariant:
\begin{align*}
    S_{QED} =& \dV\pr{\i\psibar\slashed{D}\psi-m\psibar\psi -\frac14\Fmunu F^{\mu\nu}} = \numthis\label{1:QEDaction}\\
    =& \dV\pr{\i\psibar\slashed{\partial}\psi-m\psibar\psi- g \psibar\slashed{A}\psi - \frac14\Fmunu F^{\mu\nu}}\\
    S_{QED} \rightarrow S_{QED}' =& \dV\pr{\i\psibar\slashed{\partial}\psi +\cancel{\psibar\slashed{\partial}\alpha\psi} -m\psibar\psi -g \psibar\slashed{A}\psi -\cancel{\psibar\slashed\partial\alpha\psi} -\frac14\Fmunu F^{\mu\nu}}=\\
    =& \dV\pr{\i\psibar\slashed{\partial}\psi-m\psibar\psi- g \psibar\slashed{A}\psi - \frac14\Fmunu F^{\mu\nu}} = S_{QED}
\end{align*}

\subsection{Non-Abelian Gauge Theories}
Let us now consider a theory with $N$ fermions, all with the same mass $m$, described by the spinorial fields $\psi_i(x)$ with $i=1, \dots, N$.
These $N$ fermions represent the $N$ possible charges of the same particle\footnote{For example, if $N=3$ the $3$ possible charges are the color charges of QCD, as will be shown later.} and are not to be confused with, for instance, the different possible flavors of the quarks, that describe different particles with different masses.\\
Its free action is:
\begin{equation}
    S_\psi[\psi_i(x),\psibar_i(x)] = \sum_{i=1}^N\dV\pr{\i\psibar_i\slashed\partial\psi_i - m\psibar_i\psi_i} \label{1:FreeFermionAction}
\end{equation}
From now on, the sum over $i$ (and all other repeated latin indexes) will be omitted, unless differently specified.
This action is invariant under the global transformation:
\begin{align*}
    \psi_i(x) &\rightarrow \psi_i'(x) = \U\psi_j(x) \numthis\label{1:UNTransform}\\
    \psibar_i(x) &\rightarrow \psibar_i'(x) = \psibar_j(x)\Udag
\end{align*}
if $U$ is any (constant) $N\times N$ matrix such that $UU^\dagger = U^\dagger U = \id \Leftrightarrow U^\dagger=U^{-1}$, or in other words, if $U\in\UN$.
For this reason, this transformation is also called a global $\UN$ transformation.
The phase transformation \eqref{1:PhaseTransform} is the particular case where $U=e^{-\i\alpha}\in\Uem$, that is the only Abelian (commutative) unitary group.\\
As $\UN = \SUN\otimes\Uem$ $\forall N>1$, $U\in\SUN$ instead of $U\in\UN$ can be imposed, and will be from now on, without loss of generality.\\
In an analogous way to what has been done in\secref{Sec:QED}, this invariance can be made local by implementing a proper covariant derivative, similar to \eqref{1:QEDCovDeriv}.
In order to do so, the infinitesimal $\SUN$ transformation has to be considered:
\begin{equation}
    U_{ij}(x) = \delta_{ij} + \i\theta^a(x)\pr{T^a}_{ij} + O(\theta^2) \label{1:InfinitesimalSUN}
\end{equation}
where the indices $i$ and $j$ run from $1$ to $N$ (as before) and the index $a$ runs from $1$ to $N^2-1$ (the dimension of the group $\SUN$).
The matrixes $T^a$ are the $N^2-1$ generators of $\sun$ (the Lie algebra of $\SUN$), thus they are $\NxN$ hermitean and traceless, which obey the commutation relations:
\begin{equation}
    \comm{T^a}{T^b} = \i f^{abc}T^c \label{1:StrConstSUN}
\end{equation}
where $f^{abc}$ are called \emph{structure constants} of $\sun$. The normalization of these matrices can be chosen such that they obey the condition:
\begin{equation}
    \Tr(T^aT^b) = \frac12\delta^{ab} \label{1:NormCondSUN}
\end{equation}
some examples are:
\begin{itemize}
    \item $N=2$, $T^a=\frac{\sigma^a}{2}$, with $\sigma^a$ the Pauli matrices and $f^{abc}=\varepsilon^{abc}$;
    \item $N=3$, $T^a=\frac{\lambda^a}{2}$, with $\lambda^a$ the Gell-Mann matrices.
\end{itemize}
where $\varepsilon^{abc}$ is the completely antisymmetric Levi-Civita symbol.\\
The covariant derivative, therefore, is written as:
\begin{equation}
    D_\mu \equiv \partial_\mu + \i g \A_\mu(x) \label{1:CovDeriv}
\end{equation}
where an $N\times N$ identity matrix $\id$ multiplying $\partial_\mu$ has to be understood, and $\A_\mu(x)$ is a gauge field of $\SUN$, \ie a traceless, hermitean $N\times N$ matrix, or, in other words, $\A_\mu(x)\in\sun$.\\
The covariant derivative can be written more explicitly acting on the set of spinors $\psi_i$:
\begin{equation*}
    \pr{D_\mu}_{ij}\psi_j = \partial_\mu \id_{ij}\psi_j + \i g \pr{\A_\mu(x)}_{ij}\psi_j
\end{equation*}
In order for the action to be gauge invariant, the field $\A_\mu$ must satisfy the gauge transformation property
\begin{equation}
    \A_\mu(x) \rightarrow \A_\mu'(x) = U(x)\A_\mu(x)U^\dagger(x) - \frac{\i}{g}U(x)\partial_\mu U^\dagger(x) \label{1:GaugeTransformSUN}
\end{equation}
This expression is a little more complicated than \eqref{1:QEDGaugeTransf}, due to the fact that $\A_\mu$ is now a non-commuting matrix. However if the Abelian case $\Uem$ is taken into consideration, where $U(x)=e^{-\i\alpha(x)}$, \eqref{1:QEDGaugeTransf} follows directly from \eqref{1:GaugeTransformSUN}.\\
Now, it can be easily checked that the kinetic term of the Lagrangian
\begin{equation*}
    \Lagr_K = \i\psibar_i\slashed{D}\psi_i = \i\psibar_i\slashed\partial\psi_i -g\psibar_i\slashed\A\psi_i
\end{equation*}
is gauge invariant (\ie invariant under \eqref{1:UNTransform} and \eqref{1:GaugeTransformSUN}) through direct computation:
\begin{align*}
    \Lagr_K \rightarrow \Lagr_K' =& \i\psibar_i U^\dagger\slashed\partial\pr{U\psi_i} -g\psibar_i\underbrace{U^\dagger U}_{\id}\slashed\A\underbrace{U^\dagger U}_{\id}\psi_i +\i\psibar_i\underbrace{U^\dagger U}_{\id}\pr{\slashed\partial U^\dagger}U\psi_i= \\
    =& \i\psibar_i U^\dagger\pr{\slashed\partial U}\psi_i +\underbrace{\i\psibar_i\slashed\partial\psi_i -g\psibar_i\slashed\A\psi_i}_{\Lagr_K} +\i\psibar_i\pr{\slashed\partial U^\dagger}U\psi_i= \\
    =& \Lagr_K +\i\psibar_i\gamma^\mu\pr{U^\dagger\partial_\mu U+ \partial_\mu U^\dagger U}\psi_i= \\
    =& \Lagr_K +\i\psibar_i\gamma^\mu\partial_\mu\pr{U^\dagger U}\psi_i= \\
    =& \Lagr_K +\i\psibar_i\gamma^\mu\underbrace{\partial_\mu\pr{\id}}_{=0}\psi_i = \Lagr_K
\end{align*}
Because of this fact, it is directly implied that the covariant derivative \eqref{1:CovDeriv} must transform, under a gauge transformation, in the adjoint representation:
\begin{equation}
    D_\mu \rightarrow D_\mu' = U D_\mu U^\dagger \label{1:GaugeTrCovDer}
\end{equation}
The field-strength for the field $\A_\mu$ is obtained, as for the Abelian case, through the commutator of two covariant derivatives. The computation is the same as \eqref{1:FieldStrU1}, but this time the commutator term is non-vanishing:
\begin{equation}
    \Fmunu\equiv -\frac{\i}{g}\comm{D_\mu}{D_\nu} = \partial_\mu\A_\nu -\partial_\nu\A_\mu +\i g\comm{\A_\mu}{\A_\nu} \label{1:FieldStrDef}
\end{equation}
This expression can be simplified a little by considering that $\A_\mu$ and $\Fmunu$ are elements of $\sun$, thus writing them in terms of their components \wrt the basis $T^a$:
\begin{align}
    \A_\mu(x) =& A_\mu^a(x) T^a \label{1:AmuComp}\\
    \Fmunu(x) =& \Fmunu^a(x) T^a \label{1:FmunuComp}
\end{align}
and by considering the relation \eqref{1:StrConstSUN}:
\begin{align*}
    \Fmunu^aT^a =& \pr{\partial_\mu A_\nu^a-\partial_\nu A_\mu^a}T^a +\i g\comm{A_\mu^bT^b}{A_\nu^cT^c}= \\
    =& \pr{\partial_\mu A_\nu^a-\partial_\nu A_\mu^a}T^a +\i gA_\mu^bA_\nu^c\underbrace{\comm{T^b}{T^c}}_{if^{bca}T^a}= \\
    =& \pr{\partial_\mu A_\nu^a-\partial_\nu A_\mu^a -gf^{abc}A_\mu^bA_\nu^c}T^a \\
    \Fmunu^a =& \partial_\mu A_\nu^a-\partial_\nu A_\mu^a -gf^{abc}A_\mu^bA_\nu^c \numthis\label{1:FmunuExplComp}
\end{align*}
In order to write a kinetic action for the field $\A_\mu$, a term proportional to $\Fmunu F^{\mu\nu}$, like in \eqref{1:FreeMaxwellAction}, is not enough:
because of \eqref{1:GaugeTrCovDer} and the definition \eqref{1:FieldStrDef}, it must transform as $\Fmunu F^{\mu\nu}\rightarrow U\Fmunu F^{\mu\nu} U^\dagger$, therefore it would not be gauge invariant.
In fact a gauge invariant action, called Yang-Mills action, is:
\begin{equation}
    S_{YM} = -\frac12\dV\Tr\pr{\Fmunu F^{\mu\nu}} \label{1:YMAction}
\end{equation}
because of the ciclic property of the trace.\footnote{Actually, a term proportional to $\det\pr{\Fmunu^a F^{a\mu\nu}}$ would be gauge invariant as well, but it would not be a suitable kinetic term as it would involve terms of higher order than $2$ in the components $\Fmunu^a$}
This action can be written in components, using \eqref{1:FmunuComp} and the trace property \eqref{1:NormCondSUN}:
\begin{equation}
    S_{YM} = -\frac12\dV\Tr\pr{\Fmunu^a F^{b\mu\nu}T^aT^b} = -\frac14\dV\Fmunu^a F^{a\mu\nu} \label{1:YMActionComp}
\end{equation}
Here, there are two remarks that need to be done.
The first one is that, if the gauge group is taken to be $\Uem$, the action \eqref{1:YMActionComp} reduces to \eqref{1:FreeMaxwellAction}, as $a=1$ because the group $\Uem$ has only $1$ generator.
The second one is that, if non-Abelian gauge groups are taken into consideration, this action naturally introduces self-interacting cubic and quartic terms, because the structure constants $f^{abc}$ are non-vanishing. This is, for example, the case for the group $\SU(3)$, that is used to describe gluon interaction, \ie Quantum Chromodynamics (QCD).
These self-interactions make the the Yang-Mills action interesting to be studied even alone, without any other fermionic or bosonic interacting field, as it will be shown later.
\begin{comment}
\\Everything that has been said for $\SUN$ can be also extended to $\mathit{SO}(N)$ by replacing \emph{unitary} with \emph{orthogonal} and \emph{traceless} with \emph{antisymmetric}. In fact, this discussion can be made for every compact\footnote{Compactness, \ie $\Tr{T^aT^b}$ positive defined, is required in order to have a bounded from below Hamiltonian.} group, such as the symplectic group $\mathit{Sp}(2N)$ and the five exceptional Lie groups $\mathit{G}(2)$, $\spF$, $\mathit{E}(6)$, $\mathit{E}(7)$ and $\mathit{E}(8)$.
\end{comment}

\subsection{Wick Rotation}
Up to now, actions were written in Minkowskian spacetime, where $\eta_{\mu\nu} = \diag(-1,1,1,1)$. In order to have a positive-defined metric $\eta_{\mu\nu} = \delta_{\mu\nu}$ a Wick rotation can be performed by re-defining the time coordinate to be $\tau = \i t = \i x^0$.
This rotation is not only a matter of convenience, as it is needed in perturbative QFT for the computation of (otherwise oscillating) functional integrals and because a Minkowskian lattice cannot be rigorously defined.

\section{Lattice Field Theory}
This section is based on standard quantum field theory textbooks cited before and lattice field theory textbooks \cite{Gattringer:2010zz, Montvay:1994cy, DeGrand:2006zz}.
