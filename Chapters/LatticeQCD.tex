\pagestyle{myFancy}
\chapter{QCD on the Lattice}
\section{The QCD Continuum Action}
In order to write the action of QCD on the lattice, I must first recall how the theory is formulated in the continuum.

\subsection{Spinor Fields}
Let us take into consideration a (free) quantum field theory describing a fermion, such as a quark or a lepton, in a $4$-dimensional spacetime with metric $\eta_{\mu\nu} = \diag(-1,1,1,1)$. Its action (in natural units, where $c = \hslash = 1$) can be written as:
\begin{equation}
      S_F[\psi,\psibar] = \dV \left( \i\psibar\slashed\partial\psi - m\psibar\psi \right) \label{1:FreeFermionAction}
\end{equation}
from which, upon the application of the variational principle, the Dirac equation follows:
\begin{equation}
    \left( \i\slashed\partial-m \right) \psi = 0 \label{1:DiracEq}
\end{equation}
% TODO: phase transformation

\subsection{Quantum Electrodynamics}
As the free field theory itself is non interacting, it does not provide any real-world prediction, so it is useful to write an interacting action where the spinor field is coupled, for instance, to a vector field $A_\mu$, i.e. the photon.
The action for the vector field is written in terms of its field-strength, namely:
\begin{equation}
    \Fmunu \equiv \partial_\mu A_\nu - \partial_\nu A_\mu \label{1:FieldStrU1}
\end{equation}
where two different fields $A_\mu$ and $A_\mu'$ describe the same physics if one can be obtained from another throug a gauge transformation:
\begin{align*}
    A_\mu' &= A_\mu + \partial_\mu\Lambda \numthis\label{1:GaugeTransf} \\
    \Fmunu' = \Fmunu + &\left( \partial_\mu\partial_\nu-\partial_\nu\partial_\mu \right)\Lambda = \Fmunu
\end{align*}
with $\Lambda$ being any (at least C$^2$) scalar function.\\
Thus, the free action for the vector field is:
\begin{equation}
    S_{EM} = -\frac14\dV F_{\mu\nu}F^{\mu\nu} \label{1:FreeMaxwellAction}
\end{equation}
That is also gauge invariant, i.e. invariant under \eqref{1:GaugeTransf}, as $\Fmunu$ is gauge invariant.\\
In order to write a fully covariant, gauge-invariant interacting action, the covariant derivative on the spinor has to be defined as follows:
\begin{equation}
    D_\mu\psi \equiv \left( \partial_\mu+\i e A_\mu \right)\psi \label{1:CovDeriv}
\end{equation}

