\pagestyle{myFancy}
\chapter{Computer Simulation of Pure Gauge Theories}
\section{Introduction}
Monte Carlo simulations are a powerful tool that can be used to evaluate observables in lattice field theory.
Given an observable $O$ (such as a plaquette, a Wilson or Polyakov loop, etc.), its vacuum expectation value is formally given by the functional integral
\begin{equation}
    \expval{O} = \frac1Z\int\DD[U]e^{-S[U]}O[U] \qquad \text{with} \qquad Z = \int\DD[U]e^{-S[U]}O[U] \label{2:Observable}
\end{equation}
where $\DD[U]$ is to be intended as a Haar measure.\\
This expression, however, cannot be evalued analytically except for very small lattices, therefore \eqref{2:Observable} is approximated by an average of the observable evaluated on $N$ sample gauge field configurations $\prc{U_n}$\footnote{Here the subscript $n$ distinguishes the different configurations.}, distributed according a probability density $\varpropto e^{-S[U_n]}$.\\
The expectation value is then obtained by computing the following sum for a sufficient number of configurations generated by the proper Monte Carlo algorithm(s):
\begin{equation}
    \expval{O} \simeq \frac1N\sum_{\prc{U_n}}O[U_n] \label{2:ExpValObs}
\end{equation}
Because of the probability density $\varpropto e^{-S[U_n]}$, only configurations $\prc{U_n}$ that minimize the action are \emph{good configurations}, as all other configurations are exponentially suppressed.
For this reason, generating totally random gauge fields on the lattice links is not an efficient way to evaluate \eqref{2:ExpValObs}, as most of the $\prc{U_n}$ will have a very small Boltzmann factor (the $e^{-S[U_n]}$) and expression \eqref{2:ExpValObs} will give incorrect results unless a huge number (orders of magnitude higher than what is reasonably possible) of different configurations is tried.\\
In order to avoid the evaluation of a great number of configurations that would contribute little-to-nothing to the observables' values, a sequence of configurations $\prc{U_n}$ is generated through a Markov chain process, built such that its stationary distribution minimizes the action $S[U]$. This process is usually called \emph{thermalization}.\\
Every process of this type must begin from a starting configuration, usually chosen by the user.
The main starting configurations are usually two: if the simulation begins with the fields in an ordered way (for example, all the gauge links set to the identity), the initial configuration is called \emph{cold start}, if the simulation begins with random gauge fields in every link, the initial configuration is called \emph{hot start}.
Of course, the Markov chain must have the same stationary distribution if the hot or cold start is chosen.\\
In this chapter the main Monte Carlo algorithms used to thermalize the starting configuration are presented, then the method for evaluating some observables of interest, such as plaquettes and Polyakov loops is explained.

\section{Markov Processes}
As anticipated before, a Markov chain is needed to evolve an arbitrary starting configuration $U_0$ up to a region of the configuration space with a relatively large Boltzmann factor $e^{-S}$, therefore with high probability.
A Markov process is charaterized by a probability of transitioning from a configuration $\alpha$ to another configuration $\beta$ depending only on $\alpha$ and $\beta$ and not on the history of the process, namely the previous transitions that already occourred.
In formulae, this is expressed as
\begin{equation}
    P(U_n=\beta|U_{n-1}=\alpha) = T(\beta|\alpha) \label{2:TransMatrix}
\end{equation}
That is to say that the transition matrix $T$ does not depend on the index $n$, representing the computer time.\\
Being a transition probability, the matrix $T$ must obey the following equations
\begin{align}
    0 \leq T(\beta|\alpha) \leq& 1 \qquad \forall \alpha,\beta \label{2:TPropProb} \\
    \sum_{\beta}T(\beta|\alpha) =& 1 \qquad \forall \alpha \label{2:TPropNorm}
\end{align}
where \eqref{2:TPropProb} is a consequence of $T(\beta|\alpha)$ representing a probability, and \eqref{2:TPropNorm} means that the probability of transitioning to any configuration must be $1$ (of course, the case when $\alpha=\beta$ is included as well).\\
In order for the stochastic process to not have any sink or source of probability, the following balance equation must be satisfied:
\begin{equation}
    \sum_\alpha T(\beta|\alpha)P(\alpha) = \sum_\alpha T(\alpha|\beta)P(\beta) \label{2:BalanceEq}
\end{equation}
This equation states that the probability of transitioning to the configuration $\beta$, written in the left-hand side as the sum of the transition probability from the configuration $\alpha$ weighted by the probability $P(\alpha)$ that the system is actually in that configuration, must be equal to the probability of transitioning out of the configuration $\beta$, given by the probability of finding the system in the configuration $\beta$ times the transition probability $T(\alpha|\beta)$ over all the final configurations, in the right-hand side.
Thanks to \eqref{2:TPropNorm}, the right-hand side is easily proven to be equal to $P(\beta)$.\\
A sufficient (but not necessary) condition to obey the balance equation \eqref{2:BalanceEq} is obtained by requiring that it holds true term-by-term, thus obtaining the detailed balance condition:
\begin{equation}
    T(\beta|\alpha)P(\alpha) = T(\alpha|\beta)P(\beta) \label{2:DetailedBalance}
\end{equation}
Although it is not a necessary condition, most algorithms, including the ones discussed in the following section, satisfy it.

\section{Monte Carlo Algorithms}

\subsection{Metropolis Algorithm}

\subsection{Heat Bath Algorithm}

\subsection{Overrelaxation Algorithm}
