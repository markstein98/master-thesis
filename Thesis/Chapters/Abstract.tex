\thispagestyle{empty}
\section*{Abstract}
The regularization on a Euclidean lattice, first proposed by Kenneth G. Wilson in 1974, remains the only approach to study strongly coupled, non-supersymmetric non-Abelian gauge theories (including, in particular, quantum chromodynamics: the fundamental theory of the strong nuclear interaction in the Standard Model of elementary-particle physics) from first principles.\\
While normally the theory is discretized on a four-dimensional hypercubic grid, this is not the only possible choice, and the fact that the explicit breaking of Lorentz-Poincaré symmetries due to the discretization has an impact on the approach to the continuum limit is a motivation to consider the regularization also on other, more symmetric, lattices.\\
The goal of this thesis project consists in studying Yang-Mills theories based on local SU(N) invariance on the lattice of the roots of the exceptional simple Lie group $F_4$, which is a four-dimensional body-centered cubic lattice, and the most symmetric regular lattice that exists in four dimensions.
\cleardoublepage
